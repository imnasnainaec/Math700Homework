\documentclass[11pt]{paper}
\usepackage[letterpaper]{geometry}

\usepackage{tikz-cd}
\usepackage{amsthm}

%%%%%%%%%%%%%%%%%%%%%%%%%%%%%%%%%%%%%%%%
% Basic packages
%%%%%%%%%%%%%%%%%%%%%%%%%%%%%%%%%%%%%%%%
\usepackage{amsmath,amsthm,amssymb}
\usepackage{mathtools}
\usepackage{etoolbox}
\usepackage{fancyhdr}
\usepackage{xcolor}
\usepackage{hyperref}
\usepackage{xspace}
\usepackage{comment}
\usepackage{url} % for url in bib entries
\usepackage{mathrsfs}


\theoremstyle{remark}
\newtheorem{problem}{Problem}
\newtheorem*{solution}{{\bf Solution}}

%%%%%%%%%%%%%%%%%%%%%%%%%%%%%%%%%%%%%%%%
% Acronyms
%%%%%%%%%%%%%%%%%%%%%%%%%%%%%%%%%%%%%%%%
\usepackage[acronym, shortcuts]{glossaries}

%% HERE IS HOW YOU DEFINE ACRONYMS:
\newacronym{FTA}{FTA}{Fundamental Theorem of Algebra}
\newacronym{CRT}{CRT}{Chinese Remainder Theorem}

% Make \ac robust.
\robustify{\ac}

%%%%%%%%%%%%%%%%%%%%%%%%%%%%%%%%%%%%%%%%
% Fancy page style
%%%%%%%%%%%%%%%%%%%%%%%%%%%%%%%%%%%%%%%%
\pagestyle{fancy}
\newcommand{\metadata}[2]{
  \lhead{}
  \chead{}
  \rhead{\bfseries Math 700: Linear Algebra}
  \lfoot{#1}
  \cfoot{#2}
  \rfoot{\thepage}
}
\renewcommand{\headrulewidth}{0.4pt}
\renewcommand{\footrulewidth}{0.4pt}


\newrobustcmd*{\vocab}[1]{\emph{#1}}
\newrobustcmd*{\latin}[1]{\textit{#1}}

%%%%%%%%%%%%%%%%%%%%%%%%%%%%%%%%%%%%%%%%
% Customize list enviroonments
%%%%%%%%%%%%%%%%%%%%%%%%%%%%%%%%%%%%%%%%
% package to customize three basic list environments: enumerate, itemize and description.
\usepackage{enumitem}
\setitemize{noitemsep, topsep=0pt, leftmargin=*}
\setenumerate{noitemsep, topsep=0pt, leftmargin=*}
\setdescription{noitemsep, topsep=0pt, leftmargin=*}

%%%%%%%%%%%%%%%%%%%%%%%%%%%%%%%%%%%%%%%%
%% Space between problems
\newrobustcmd*{\probskip}{\vskip1cm}

%% This is the Homework LaTeX template.  Use this file to fill in your solutions. 
%%
%% Notes: 
%%    1. Write your answers inside a \begin{solution}...\end{solution} environment.
%%
%%    2. If you will use references, insert bibtex reference entries in the file
%%       Math700.bib.  (Create that file if it doesn't yet exist.)
%%
%%    3. If you will use acronyms, please define them in the macros.tex file.
%%
%%    4. Please try to check that your file compiles:
%%
%%       Mac OS X users: you might try MacTeX. 
%%       Windows users: you might try proTeXt. 
%%       Linux users: most come with TeX; otherwise do a full install of TeXLive.
%%
%%       There is a Makefile in this directory, so on Linux you could just 
%%       enter `make` to compile all the Homework*.tex files at once.
%%
%%    5. Please don't hesitate to inform the prof if you have trouble, or open
%%       a ``New issue'' or create a new ``Wiki page'' on GitHub.  Otherwise,
%%       send an email to williamdemeo@gmail.com.
%%
%%    6. It will probably be hard to keep everyone's notation consistent.
%%       For the most basic symbols, we should have some conventions and use
%%       LaTeX macros to keep the conventions consistent and easy to remember.
%%       For example, to denote an algebra,
         \newcommand\alg[1]{\ensuremath{\mathbf{#1}}}
         \newcommand{\<}{\ensuremath{\langle}}
         \renewcommand{\>}{\ensuremath{\rangle}}
%%       So, an algebra in LaTeX is typed as $\alg{A} = \<A, F\>$.
%%       Similarly, for a field, let's use:
         \newcommand\fld[1]{\ensuremath{\mathbb{#1}}}
%%       So, a field in LaTeX is typed as $\fld{F}$.

%%
%%    7. Replace these names with yours!!!
         \metadata{Emmy and Emil}{Homework 1 -- 2014/01/13}
         \author{Emil Artin and Emmy Noether}
%%
%%    8. Update the title and date as appropriate.
         \title{Homework 1}
         \date{January 13, 2014}

\begin{document}

\maketitle


%%%%%%%%%%%%%%%%%%%%%%%%%%%%%%%%%%%%%%%%%%%%%%%%%%%%%%%%%%%%%%%%%%%%%%%%%%%%
\begin{problem}[Golan 12]
For a field $\fld{F} = \<F,+,\cdot, -, 0, 1\>$, 
show that the function $a \mapsto a^{-1}$ is a 
permutation of the set $F \setminus \{0_F\}$.
\end{problem}
\smallskip
\begin{solution}

%%TODO: Write your solution here.

\end{solution}
\probskip




%%%%%%%%%%%%%%%%%%%%%%%%%%%%%%%%%%%%%%%%%%%%%%%%%%%%%%%%%%%%%%%%%%%%%%%%%%%%
\begin{problem}[Golan 16]
Let $z_1$, $z_2$, and $z_3$ be complex numbers satisfying 
$|z_i| = 1$ for $i = 1, 2, 3$. Show that 
$|z_1 z_2 + z_1 z_3 + z_2 z_3 | = |z_1 + z_2 + z_3|$.
\end{problem}
\smallskip
\begin{solution}

%% Write your solution here.

\end{solution}
\probskip




%%%%%%%%%%%%%%%%%%%%%%%%%%%%%%%%%%%%%%%%%%%%%%%%%%%%%%%%%%%%%%%%%%%%%%%%%%%%
\begin{problem}[Golan 22 {\it Abel's inequality}] 
Let $z_1, \dots, z_n$ be a list of complex
numbers and, for each $1 \leq k \leq n$, 
let $s_k = \sum_{i=1}^k z_i$. For real numbers
$a_1, \dots, a_n$ satisfying 
$a_1 \geq a_2 \geq \cdots \geq a_n \geq 0$, 
show that
\begin{equation}
\label{eq:Abels}  
\left| \sum_{i=1}^k a_i z_i \right| 
\leq a_1 \left( \max_{1 \leq k \leq n} |s_k|\right).
\end{equation}
\end{problem}
\smallskip
\begin{solution}

%% Write your solution here.

\end{solution}
\probskip



%%%%%%%%%%%%%%%%%%%%%%%%%%%%%%%%%%%%%%%%%%%%%%%%%%%%%%%%%%%%%%%%%%%%%%%%%%%%
\begin{problem}[Golan 24]
If $p$ is a prime positive integer, find all subfields of $GF(p)$.
\end{problem}


\begin{solution}

%% Write your solution here.

\end{solution}
\probskip
%%%%%%%%%%%%%%%%%%%%%%%%%%%%%%%%%%%%%%%%%%%%%%%%%%%%%%%%%%%%%%%%%%%%%%%%%%%%
\begin{problem}
Write down the definition of a \emph{module} as a (universal) algebra, 
$\alg{M} = \< M, F\>$.  That is, describe the set $F$ of operations and 
give the conditions that they should satisfy in order for $\alg{M}$ to 
agree with the classical definition of a module over a ring.\\[4pt]
[{\it Hint:} Let $\alg{R} = \<R, +, \cdot, -, 0, 1\>$ be a ring and,
  for each $r\in R$, define a scalar multiply operation $f_r \in F$.]
\end{problem}

\begin{solution}

%%TODO: Write your solution here.

\end{solution}
\probskip



%%%%%%%%%%%%%%%%%%%%%%%%%%%%%%%%%%%%%%%%%%%%%%%%%%%%%%%%%%%%%%%%%%%%%%%%%%%%
\begin{problem}
Let $\alg{R} = \<R, +, \cdot, -, 0, 1\>$ be a ring.  
\begin{enumerate}
\item Define \emph{left ideal} of $\alg{R}$.
\item Let $\mathscr{A} = \{A_i : i \in \mathscr{I}\}$ be a family of left ideals
of $\alg{R}$.  Prove that $\bigcap \mathscr{A}$ is a left ideal.
\end{enumerate}
\end{problem}
\smallskip
\begin{solution}

%%TODO: Write your solution here.

\end{solution}
\probskip


%%%%%%%%%%%%%%%%%%%%%%%%%%%%%%%%%%%%%%%%%%%%%%%%%%%%%%%%%%%%%%%%%%%%%%%%%%%%
\begin{problem}
Let $\alg{R}$ be a ring and fix $a, b \in R$.  Prove that if $1 - ba$ is left
invertible, then $1 - ab$ is also left invertible.  What is the inverse?\\[4pt]
[{\it Hint:} Consider the left ideal $R(1 - ab)$.  It contains the left ideal
  $Rb(1-ab) = Rb$ and therefore contains 1. Verify these statements, then
  try to compute the inverse of $1-ab$, as follow.  (Ask for more hints as needed.)] 
\end{problem}
\smallskip
\begin{solution}

%%TODO: Write your solution here.

\end{solution}
\probskip


%% If you will use references, add your refs to the Math700.bib file.
%% and then uncomment the following lines.
%% \bibliographystyle{plain}
%% \bibliography{Math700}

\end{document}
